\definecolor{HartmannDarkBlue}{RGB}{5,99,255}				% Definition des dunklen Blaus �ber RGB-Farben
\definecolor{HartmannLightBlue}{RGB}{167,200,255}			% Definition des hellen Blaus �ber RGB-Farben
\definecolor{HartmannBlue}{RGB}{5,99,255}					% Definition des Dr. Hartmann blaus �ber RGB-Farben

\TabellenKopfHintergrundfarbe{HartmannDarkBlue}				% Hintergundfarbe f�r die Tabellenk�pfe
\TabellenZweiteKopfHintergrundfarbe{HartmannLightBlue}		% Hintergundfarbe f�r die Tabellenk�pfe
\TabellenKopfSchriftfarbe{white}							% Schriftfarbe f�r die Tabellenk�pfe
\TabellenHintergrund{lightgray}								% Hintergrund der colorierten Tabellenzellen

\TitelseiteLogo{\mpathtologos hartmann_ct.jpg}				% Bild auf der Titelseite, auskommentieren wenn kein Bild gew�nscht
\FusszeilenLogo{\mpathtologos hartmann_ct.jpg}

% nicht sch�n!, aber keine andere L�sung gefunden, die beiden Werte hinter den zwei nullen m�ssen durch die entsprechende Gr��e des Logos erstetzt werden (erst Breite dann H�he)
\FusszeilenLogoGroesse{\includegraphics[bb = 0 0 250 135 ,height=1.3cm]{\mfusszeilenlogo}}
%\FusszeilenLogoGroesse{\includegraphics[bb = 0 15 1800 1100 ,height=1.3cm]{\mfusszeilenlogo}}

\TitlePageFile{titlepagehartmann} 							% Name der gew�nschten Titelseitendatei
\ContentsPageFile{inhaltsverzeichnishartmann}				% Name der gew�nschten Inhaltsverzeichnisdatei
\HeaderFooterFile{\mpathtoheads hartmannct}					% Pfad zur gew�nschten datei welche die Kopf und Fusszeile festlegt (nohead wenn keine Kopf-bzw. Fusszeilen gew�nscht)