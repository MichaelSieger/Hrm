\definecolor{BayernFMBlue}{RGB}{0,145,254}					% Definition des Bayern FM Blaus �ber RGB-Farben
\definecolor{BayernFMOrange}{RGB}{254,125,33}				% Definition des Bayern FM Oranges �ber RGB-Farben
\definecolor{BayernFMGray}{RGB}{98,98,98}					% Definition des Bayern FM Graus �ber RGB-Farben
\definecolor{SysHygBlue}{RGB}{0,159,224}					% Definition des System Hygiene Blaus �ber RGB-Farben

\TabellenKopfHintergrundfarbe{BayernFMBlue}					% Hintergundfarbe f�r die Tabellenk�pfe
\TabellenZweiteKopfHintergrundfarbe{BayernFMGray}			% Hintergundfarbe f�r die Tabellenk�pfe
\TabellenKopfSchriftfarbe{white}							% Schriftfarbe f�r die Tabellenk�pfe
\TabellenHintergrund{lightgray}								% Hintergrund der colorierten Tabellenzellen

\TitelseiteLogo{\mpathtologos bayernfm.jpg}			% Bild auf der Titelseite, auskommentieren wenn kein Bild gew�nscht
\FusszeilenLogo{\mpathtologos bayernfm.jpg}

% nicht sch�n!, aber keine andere L�sung gefunden, die beiden Werte hinter den zwei nullen m�ssen durch die entsprechende Gr��e des Logos erstetzt werden (erst Breite dann H�he)
\FusszeilenLogoGroesse{\includegraphics[bb = 10 0 1100 700 ,height=1.3cm]{\mfusszeilenlogo}}

\TitlePageFile{titlepage} 									% Name der gew�nschten Titelseitendatei
\ContentsPageFile{inhaltsverzeichnis} 						% Name der gew�nschten Inhaltsverzeichnisdatei
\HeaderFooterFile{\mpathtoheads bayernfm}				% Pfad zur gew�nschten datei welche die Kopf und Fusszeile festlegt (nohead wenn keine Kopf-bzw. Fusszeilen gew�nscht)