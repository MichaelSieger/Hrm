\definecolor{HartmannDarkGreen}{RGB}{23,169,79}				% Definition des dunklen Gr�ns �ber RGB-Farben
\definecolor{HartmannLightGreen}{RGB}{150,179,161}			% Definition des hellen Gr�ns �ber RGB-Farben
\definecolor{HartmannGreen}{RGB}{23,169,79}					% Definition des Dr. Hartmann gr�ns �ber RGB-Farben

\TabellenKopfHintergrundfarbe{HartmannDarkGreen}			% Hintergundfarbe f�r die Tabellenk�pfe
\TabellenZweiteKopfHintergrundfarbe{HartmannLightGreen}		% Hintergundfarbe f�r die Tabellenk�pfe
\TabellenKopfSchriftfarbe{white}							% Schriftfarbe f�r die Tabellenk�pfe
\TabellenHintergrund{lightgray}								% Hintergrund der colorierten Tabellenzellen

\TitelseiteLogo{\mpathtologos Service_HARTMANN.jpg}			% Bild auf der Titelseite, auskommentieren wenn kein Bild gew�nscht
\FusszeilenLogo{\mpathtologos Service_HARTMANN.jpg}

% nicht sch�n!, aber keine andere L�sung gefunden, die beiden Werte hinter den zwei nullen m�ssen durch die entsprechende Gr��e des Logos erstetzt werden (erst Breite dann H�he)
\FusszeilenLogoGroesse{\includegraphics[bb = 0 15 1800 1100 ,height=1.3cm]{\mfusszeilenlogo}}

\TitlePageFile{titlepagehartmann} 							% Name der gew�nschten Titelseitendatei
\ContentsPageFile{inhaltsverzeichnishartmann}				% Name der gew�nschten Inhaltsverzeichnisdatei
\HeaderFooterFile{\mpathtoheads hartmannservice}			% Pfad zur gew�nschten datei welche die Kopf und Fusszeile festlegt (nohead wenn keine Kopf-bzw. Fusszeilen gew�nscht)