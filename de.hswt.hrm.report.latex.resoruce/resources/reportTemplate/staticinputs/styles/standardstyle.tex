\definecolor{MyDarkBlue}{RGB}{0,15,89}						% Definition des dunklen Blaus �ber RGB-Farben
\definecolor{MyLightBlue}{RGB}{148,164,200}					% Definition des hellen Blaus �ber RGB-Farben
\definecolor{SysHygBlue}{RGB}{0,159,224}					% Definition des System Hygiene Blaus �ber RGB-Farben

\TabellenKopfHintergrundfarbe{MyDarkBlue}					% Hintergundfarbe f�r die Tabellenk�pfe
\TabellenZweiteKopfHintergrundfarbe{MyLightBlue}			% Hintergundfarbe f�r die Tabellenk�pfe
\TabellenKopfSchriftfarbe{white}							% Schriftfarbe f�r die Tabellenk�pfe
\TabellenHintergrund{lightgray}								% Hintergrund der colorierten Tabellenzellen

\TitelseiteLogo{\mpathtologos logo.jpg}						% Bild auf der Titelseite, auskommentieren wenn kein Bild gew�nscht
\FusszeilenLogo{\mpathtologos logo.jpg}

% nicht sch�n!, aber keine andere L�sung gefunden, die beiden Werte hinter den zwei nullen m�ssen durch die entsprechende Gr��e des Logos erstetzt werden (erst Breite dann H�he)
\FusszeilenLogoGroesse{\includegraphics[bb = 0 0 170 75 ,height=1.3cm]{\mfusszeilenlogo}}

\TitlePageFile{titlepage} 									% Name der gew�nschten Titelseitendatei
\ContentsPageFile{inhaltsverzeichnis} 						% Name der gew�nschten Inhaltsverzeichnisdatei
\HeaderFooterFile{\mpathtoheads standardhead}				% Pfad zur gew�nschten datei welche die Kopf und Fusszeile festlegt (nohead wenn keine Kopf-bzw. Fusszeilen gew�nscht)