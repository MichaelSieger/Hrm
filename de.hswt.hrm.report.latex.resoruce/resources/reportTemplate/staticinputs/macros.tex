% macro definitions

\newcommand{\TitlePageFile}[1]{\newcommand{\mtitlepagefile}{#1}}              % header und footer file
\newcommand{\ContentsPageFile}[1]{\newcommand{\mcontentspagefile}{#1}}        % header und footer file
\newcommand{\HeaderFooterFile}[1]{\newcommand{\mheaderfooterfile}{#1}}        % header und footer file

\newcommand{\TabellenKopfHintergrundfarbe}[1]{\newcommand{\mtabellenkopfhintergrundfarbe}{#1}}                  % Tabellenkopfhintergrundfarbe
\newcommand{\TabellenZweiteKopfHintergrundfarbe}[1]{\newcommand{\mtabellenzweitekopfhintergrundfarbe}{#1}}      % Tabellenkopfhintergrundfarbe
\newcommand{\TabellenKopfSchriftfarbe}[1]{\newcommand{\mtabellenkopfschriftfarbe}{#1}}                          % Tabellenkopfschriftfarbe
\newcommand{\TabellenHintergrund}[1]{\newcommand{\mtabellenhintergrund}{#1}}                                    % Tabellenhintergrund

\newcommand{\TitelseiteLogo}[1]{\newcommand{\mtitelseitelogo}{#1}}
\newcommand{\FusszeilenLogo}[1]{\newcommand{\mfusszeilenlogo}{#1}}
\newcommand{\FusszeilenLogoGroesse}[1]{\newcommand{\mfusszeilenlogogroesse}{#1}}

\newcommand{\BioGridGroesse}[1]{\newcommand{\mbiogridgroesse}{#1}}                                              % Die Ausdehnung bzw der Maßstab des Biogrids 
\newcommand{\BioGridBreite}[1]{\newcommand{\mbiogridbreite}{#1}}                                                % Breite des Biogrids (Anzahl Teilflächen)
\newcommand{\BioGridHoehe}[1]{\newcommand{\mbiogridhoehe}{#1}}                                                  % Höhe des Biogrids (Anzahl Teilflächen)
\newcommand{\PhysGridGroesse}[1]{\newcommand{\mphysgridgroesse}{#1}}                                            % Die Ausdehnung bzw der Maßstab des Physgrids 
\newcommand{\PhysGridBreite}[1]{\newcommand{\mphysgridbreite}{#1}}                                              % Breite des Physgrids (Anzahl Teilflächen)
\newcommand{\PhysGridHohe}[1]{\newcommand{\mphysgridhoehe}{#1}}                                                 % Höhe des Physgrids (Anzahl Teilflächen)

\newcommand{\SymbolFarbeX}[1]{\newcommand{\msymbolfarbeX}{#1}}
\newcommand{\SymbolFarbeA}[1]{\newcommand{\msymbolfarbeA}{#1}}
\newcommand{\SymbolFarbeB}[1]{\newcommand{\msymbolfarbeB}{#1}}
\newcommand{\SymbolFarbeC}[1]{\newcommand{\msymbolfarbeC}{#1}}
\newcommand{\SymbolFarbeD}[1]{\newcommand{\msymbolfarbeD}{#1}}
\newcommand{\SymbolFarbeE}[1]{\newcommand{\msymbolfarbeE}{#1}}

\newcommand{\SymbolBackground}[1]{\newcommand{\msymbolbackground}{#1}}                                                                  % temporäre variable für die Hintergrundfarbe der Symbole
\SymbolBackground{white}

\newcommand{\PathToIncludes}[1]{\newcommand{\mpathtoincludes}{#1}}                                                                                      % relativer Pfad zu den Includedateien
\newcommand{\PathToSymbols}[1]{\newcommand{\mpathtosymbols}{#1}}                                                                                        % relativer Pfad zu den Parameter Symbolen
\newcommand{\PathToGrids}[1]{\newcommand{\mpathtogrids}{#1}}                                                                                            % relativer Pfad zu den Parameter Gridfiles
\newcommand{\PathToTitlePages}[1]{\newcommand{\mpathtotitlepages}{#1}}                                                                          % relativer Pfad zu den Kopf und Titelseiten
\newcommand{\PathToContentPages}[1]{\newcommand{\mpathtocontentpages}{#1}}                                                                      % relativer Pfad zu den Kopf und Inahltsverzeichnissen
\newcommand{\PathToStyles}[1]{\newcommand{\mpathtostyles}{#1}}                                                                      % relativer Pfad zu den Kopf und Inahltsverzeichnissen
\newcommand{\PathToHeads}[1]{\newcommand{\mpathtoheads}{#1}}                                                                                            % relativer Pfad zu den Kopf und Fusszeilendateien
\newcommand{\PathToImages}[1]{\newcommand{\mpathtoimages}{#1}}                                                                                          % relativer Pfad zu den Bildateien
\newcommand{\PathToLogos}[1]{\newcommand{\mpathtologos}{#1}}                                                                                            % relativer Pfad zu den Logos

% Titelseite Makros
\newcommand{\TitelseiteObjektName}[1]{\newcommand{\mtitelseiteobjektname}{#1}}                  % Titelseite Objekt Name
\newcommand{\TitelseiteObjektFoto}[1]{\newcommand{\mtitelseiteobjektfoto}{#1}}                  % Titelseite Objekt Foto

% Auftraggeber Makros
\newcommand{\Auftraggeber}[1]{\newcommand{\mauftraggeber}{#1}}                                  % AuftraggeberName
\newcommand{\AuftraggeberStrasse}[1]{\newcommand{\mauftraggeberstrasse}{#1}}                    % AuftraggeberStraße
\newcommand{\AuftraggeberStadt}[1]{\newcommand{\mauftraggeberstadt}{#1}}                        % AuftraggeberStadt

% Auftragnehmer Makros
\newcommand{\Auftragnehmer}[1]{\newcommand{\mauftragnehmer}{#1}}                                % AuftragnehmerName
\newcommand{\AuftragnehmerStrasse}[1]{\newcommand{\mauftragnehmerstrasse}{#1}}                  % AuftragnehmerStraße
\newcommand{\AuftragnehmerStadt}[1]{\newcommand{\mauftragnehmerstadt}{#1}}                      % AuftragnehmerStadt

% Auftragnehmer Makros
\newcommand{\Pruefer}[1]{\newcommand{\mpruefer}{#1}}                                            % PrüferName
\newcommand{\PrueferStrasse}[1]{\newcommand{\mprueferstrasse}{#1}}                              % PrüferStraße
\newcommand{\PrueferStadt}[1]{\newcommand{\mprueferstadt}{#1}}                                  % PrüferStadt

% Objekt (Gebäude) Makros 
\newcommand{\Objekt}[1]{\newcommand{\mobjekt}{#1}}                                              % ObjektName
\newcommand{\ObjektStrasse}[1]{\newcommand{\mobjektstrasse}{#1}}                                % ObjektStraße
\newcommand{\ObjektStadt}[1]{\newcommand{\mobjektstadt}{#1}}                                    % ObjektStadt

% Datums Makros
\newcommand{\DatumInspektion}[1]{\newcommand{\mdatuminspektion}{#1}}                            % Datum der Inspektion
\newcommand{\DatumBericht}[1]{\newcommand{\mdatumbericht}{#1}}                                  % Datum des Berichts

% Anlagendaten Makros
\newcommand{\Anlage}[1]{\newcommand{\manlage}{#1}}                                                                              % Anlage
\newcommand{\AnlagenStandort}[1]{\newcommand{\manlagenstandort}{#1}}                                    % Anlagenstandort
\newcommand{\AnlagenBereich}[1]{\newcommand{\manlagenbereich}{#1}}                                              % Analagenbereich bzw. versorgter Bereich

\newcommand{\AnlagenFoto}[1]{\newcommand{\manlagenfoto}{#1}}                                                    % Anlagenfoto
\newcommand{\AnlagenHersteller}[1]{\newcommand{\manlagenhersteller}{#1}}                                % Anlagenhersteller
\newcommand{\AnlagenTyp}[1]{\newcommand{\manlagentyp}{#1}}                                                              % Anlagentyp
\newcommand{\AnlagenBaujahr}[1]{\newcommand{\manlagenbaujahr}{#1}}                                              % Anlagenbaujahr
\newcommand{\AnlagenLuftleistung}[1]{\newcommand{\manlagenluftleistung}{#1}}                    % Anlagenluftleistung
\newcommand{\AnlagenMotorleistung}[1]{\newcommand{\manlagenmotorleistung}{#1}}                  % Anlganemotorleistung
\newcommand{\AnlagenMotordrehzahl}[1]{\newcommand{\manlagenmotordrehzahl}{#1}}                  % Anlagendrehzahl
\newcommand{\AnlagenNennstrom}[1]{\newcommand{\manlagennennstrom}{#1}}                                  % Anlagennennstrom
\newcommand{\AnlagenSpannung}[1]{\newcommand{\manlagenspannung}{#1}}                                    % Anlagenspannung


% Makros für die Breite der zweispaltigen Tabellenspalten
\newcommand{\TabellenZweiSpaltenBreiteErste}[1]{                                % Spaltenbreite der ersten Spalte bei zweispaltigen Tabellen
        \newlength{\tabellenzweispaltenbreiteerste}
        \setlength{\tabellenzweispaltenbreiteerste}{#1}
}                       
\newcommand{\TabellenZweiSpaltenBreiteZweite}[1]{                               % Spaltenbreite der zweiten Spalte bei zweispaltigen Tabellen
        \newlength{\tabellenzweispaltenbreitezweite}
        \setlength{\tabellenzweispaltenbreitezweite}{#1}
}               

% Makros für die Breite der biologischen Parameter Tabellenspalten
\newcommand{\TabelleBioParameterA}[1]{\newlength{\tabellebioparameterA}\setlength{\tabellebioparameterA}{#1}}
\newcommand{\TabelleBioParameterB}[1]{\newlength{\tabellebioparameterB}\setlength{\tabellebioparameterB}{#1}}
\newcommand{\TabelleBioParameterC}[1]{\newlength{\tabellebioparameterC}\setlength{\tabellebioparameterC}{#1}}
\newcommand{\TabelleBioParameterD}[1]{\newlength{\tabellebioparameterD}\setlength{\tabellebioparameterD}{#1}}
\newcommand{\TabelleBioParameterE}[1]{\newlength{\tabellebioparameterE}\setlength{\tabellebioparameterE}{#1}}

% Makros für die Breite der biologischen Notenberechnung Tabellenspalten
\newcommand{\TabelleBioNotenA}[1]{\newlength{\tabellebionotenA}\setlength{\tabellebionotenA}{#1}}
\newcommand{\TabelleBioNotenB}[1]{\newlength{\tabellebionotenB}\setlength{\tabellebionotenB}{#1}}
\newcommand{\TabelleBioNotenC}[1]{\newlength{\tabellebionotenC}\setlength{\tabellebionotenC}{#1}}
\newcommand{\TabelleBioNotenD}[1]{\newlength{\tabellebionotenD}\setlength{\tabellebionotenD}{#1}}
\newcommand{\TabelleBioNotenE}[1]{\newlength{\tabellebionotenE}\setlength{\tabellebionotenE}{#1}}
\newcommand{\TabelleBioNotenF}[1]{\newlength{\tabellebionotenF}\setlength{\tabellebionotenF}{#1}}

% Makros für die Breite der physikalischen Parameter Tabellenspalten
\newcommand{\TabellePhysParameterA}[1]{\newlength{\tabellephysparameterA}\setlength{\tabellephysparameterA}{#1}}
\newcommand{\TabellePhysParameterB}[1]{\newlength{\tabellephysparameterB}\setlength{\tabellephysparameterB}{#1}}
\newcommand{\TabellePhysParameterC}[1]{\newlength{\tabellephysparameterC}\setlength{\tabellephysparameterC}{#1}}
\newcommand{\TabellePhysParameterD}[1]{\newlength{\tabellephysparameterD}\setlength{\tabellephysparameterD}{#1}}

% Makros für die Breite der physikalischen Notenberechnung Tabellenspalten
\newcommand{\TabellePhysNotenA}[1]{\newlength{\tabellephysnotenA}\setlength{\tabellephysnotenA}{#1}}
\newcommand{\TabellePhysNotenB}[1]{\newlength{\tabellephysnotenB}\setlength{\tabellephysnotenB}{#1}}
\newcommand{\TabellePhysNotenC}[1]{\newlength{\tabellephysnotenC}\setlength{\tabellephysnotenC}{#1}}
\newcommand{\TabellePhysNotenD}[1]{\newlength{\tabellephysnotenD}\setlength{\tabellephysnotenD}{#1}}
\newcommand{\TabellePhysNotenE}[1]{\newlength{\tabellephysnotenE}\setlength{\tabellephysnotenE}{#1}}
\newcommand{\TabellePhysNotenF}[1]{\newlength{\tabellephysnotenF}\setlength{\tabellephysnotenF}{#1}}

% Makros für die Breite der physikalischen Soll-Ist-Maßnahmen-Übersichttabellenspalten
\newcommand{\TabellePhysSollIstMnA}[1]{\newlength{\tabellephyssollistmnA}\setlength{\tabellephyssollistmnA}{#1}}
\newcommand{\TabellePhysSollIstMnB}[1]{\newlength{\tabellephyssollistmnB}\setlength{\tabellephyssollistmnB}{#1}}
\newcommand{\TabellePhysSollIstMnC}[1]{\newlength{\tabellephyssollistmnC}\setlength{\tabellephyssollistmnC}{#1}}

% Makros für die Breite der physikalischen Soll-Ist Tabellenspalten
\newcommand{\TabellePhysSollIstA}[1]{\newlength{\tabellephyssollistA}\setlength{\tabellephyssollistA}{#1}}
\newcommand{\TabellePhysSollIstB}[1]{\newlength{\tabellephyssollistB}\setlength{\tabellephyssollistB}{#1}}
\newcommand{\TabellePhysSollIstC}[1]{\newlength{\tabellephyssollistC}\setlength{\tabellephyssollistC}{#1}}
\newcommand{\TabellePhysSollIstD}[1]{\newlength{\tabellephyssollistD}\setlength{\tabellephyssollistD}{#1}}
\newcommand{\TabellePhysSollIstE}[1]{\newlength{\tabellephyssollistE}\setlength{\tabellephyssollistE}{#1}}
\newcommand{\TabellePhysSollIstF}[1]{\newlength{\tabellephyssollistF}\setlength{\tabellephyssollistF}{#1}}
\newcommand{\TabellePhysSollIstG}[1]{\newlength{\tabellephyssollistG}\setlength{\tabellephyssollistG}{#1}}
\newcommand{\TabellePhysSollIstH}[1]{\newlength{\tabellephyssollistH}\setlength{\tabellephyssollistH}{#1}}

\newcommand{\beginmyitemize}{\begin{itemize}[noitemsep, nolistsep, fullwidth, leftmargin=0pt, rightmargin=0pt, topsep=-1cm]}
